\documentclass[a4paper,fleqn,notitlepage]{scrartcl}

\usepackage{caption}
\captionsetup[figure]{font=small,skip=0pt}

\usepackage{float}
\floatstyle{boxed} 
\restylefloat{figure}

\title{Reading Response}
\subtitle{\normalfont ``How to make {\it ad-hoc} polymorphism less {\it ad hoc}'' by Wadler \& Blott}
\author{Ross Bayer}
\date{February 26, 2016}

\begin{document}

\maketitle

%%%%%%%%%%%%%%%%%%%%%%%%%%%%%%%%%%%%%%%%%%%%%%%%%%%%%%%%%%%%%%%%%%%%%%%%%%%%%%%%

In their paper ``How to make \textit{ad-hoc} polymorphism less \textit{ad hoc}'', 
Wadler and Blott propose a new, general purpose method to introduce adhoc
polymorphism to an existing Hindley-Milner type system. Wadler and Blott were
unsatisfied with the existing strategies to handle ad-hoc polymorphism present
in both Standard ML and the Miranda language. They propose a new language level
construct known as a \textit{type class} which they demonstrate to be more
flexible than any previously proposed implementation strategy.
\textit{Type classes} were developed along-side the Haskell language, which 
later popularized the idea.

\textit{Type classes} represent a type-level construct in the language that
defines a set of associated functions for one or more unspecified mono-types. An
\textit{instance} of a \textit{type class} provides concrete definitions of
each associated function for the specific mono-type(s). For example, the \verb|Eq|
\textit{type class} presented in the paper, defines the \verb|(==)| function,
used for equality testing. An implementation of the \verb|Eq| \textit{type class}
for the mono-type \verb|Int| defines concretely how two integers should be
tested for equality. The \verb|Eq| \textit{type class} is defined in Figure 1.

\begin{figure}
\caption{Definition of the \protect\texttt{Eq} class}
\begin{verbatim}
    
    class Eq a where
        (==) :: a -> a -> Bool
    
    instance Eq Int where
        x == y = eqInt x y

\end{verbatim}
\end{figure}

For those with experience in Haskell, the \verb|Eq| class should be
familiar.

According to Wadler and Blott, a \textit{type class} can then be used to define
functions that are polymorphic over a \textit{type class}, such as the \verb|member|
function in Figure 2. The \verb|member| function determines if a given element
if a member of the given list, with the constraint that the mono-type \verb|a| is
a member of the \textit{type class} \verb|Eq|.

\begin{figure}
\caption{Definition of the \protect\texttt{member} function}
\begin{verbatim}
    
    member :: Eq a => [a] -> a -> Bool
    member []     y = False
    member (x:xs) y = (x == y) \/ member xs y

\end{verbatim}
\end{figure}

Wadler and Blott also describe a translation step, which allows \textit{type classes}
to be introduced to an existing Hindley-Milner type system. It translates a
\textit{type class} into some form of dictionary that contains the concrete
definitions of the associated functions. For example, Figure 3 demonstrates how the
\verb|Eq| \textit{type class} would be translated and used in the definition of the
\verb|member| function.

\begin{figure}
\caption{Translation of the \protect\texttt{Eq} class and \protect\texttt{member} function}
\begin{verbatim}
    
    data EqD a = EqDict (a -> a -> Bool)
    
    eq (EqDict a) = a
    
    eqDInt :: EqD Int
    eqDInt =  EqDict eqInt
    
    eqDChar :: EqD Char
    eqDChar =  EqDict eqChar
    
    member :: EqD a -> [a] -> a -> Bool
    member eqDa [] y = False
    member eqDa (x:xs) y = ((eq eqDa) x y) \/ (member eqDa xs y)

\end{verbatim}
\end{figure}

Furthermore, Wadler and Blott's \textit{type classes} allow for subclassing.
Subclassing introduces constraints on \textit{type classes}, allowing programmers
to express natural properties of a \textit{type class} as a member of already
existing \textit{type classes}. The example provided was the \verb|Num| class,
which is a subclass of the \verb|Eq| class, as it can be assumed that any member
of the \verb|Num| class would some form of numeric value, which can be compared
for equality. Figure 4 demonstrates subclassing.

\begin{figure}
\caption{\protect\texttt{Num} class subclassing the \protect\texttt{Eq} class}
\begin{verbatim}
    
    class Eq a => Num a where
        (+)    :: a -> a -> a
        (*)    :: a -> a -> a
        negate :: a -> a

\end{verbatim}
\end{figure}

\end{document}
