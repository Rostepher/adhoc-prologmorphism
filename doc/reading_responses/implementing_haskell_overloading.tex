\documentclass[a4paper,fleqn,notitlepage]{scrartcl}

\usepackage{caption}
\captionsetup[figure]{font=small,skip=0pt}

\usepackage{float}
\floatstyle{boxed} 
\restylefloat{figure}

\title{Reading Response}
\subtitle{\normalfont ``Implementing Haskell Overloading'' by Lennnart Augustsson}
\author{Ross Bayer}
\date{March 4, 2016}

\begin{document}

\maketitle

%%%%%%%%%%%%%%%%%%%%%%%%%%%%%%%%%%%%%%%%%%%%%%%%%%%%%%%%%%%%%%%%%%%%%%%%%%%%%%%%

Lennart Augustsson's paper, ``Implementing Haskell Overloading'' describes a
number of optimizations for the efficient implementation of adhoc-polymorphism
in the context of the Haskell programming language. The paper also describes a 
few other miscellaneous optimizations for improving various aspects of the
language. This paper builds on Wadler and Blott's ``Making adhoc-polymorphism
less ad hoc'' by modifying the type class implementation previously proposed.
All the optimizations mentioned are not the author's creations, but rather
this paper is a compilation of existing optimization strategies with
modifications for the Haskell programming language, that when taken together
can provide a considerable speed increase.

Type classes can be implemented as a simple translation step that transforms
functions with type class constraints into functions with additional explicit
arguments for passing some data structure that contains the concrete
implementations of a type classes associated functions for a given type.
The data structure used can be a tuple of $n$ elements, where $n$ is the number
of functions in the type class, or a record (also called a dictionary) with
fields for each function. Helper functions to access each function implementation
are necessary. This paper assumes that these ``dictionaries'' are implemented as
tuples of $n$ elements, rather than records.

Augustsson points out that a naive translation step is inefficient. Passing around
dictionaries can be costly and lead to performance degradation. He asserts that
a compiler can utilize a number of optimizations to improve performance. The
optimizations most relevant to the direct implementation of overloading are
modifying the representation of the dictionaries and specialization.

Augustsson provides a table at the end of this paper that presents execution
times for the various optimizations described when run on a few benchmarks.
The conclusions that can be drawn from the measurements is that if the various
strategies described in this paper are utilized then the execution of the
program can be improved somewhere between moderately and dramatically. The table
is incomplete however, with a large portion of combinations left untested.
As a reader, I am left wondering if the tests are statistically significant?
What hardware were the tests executed on? In general, the results seem mostly
informal and only useful as a general notion of possible performance increases.

Overall this paper was noteworthy, not because the ideas presented in it were
new and novel, but rather because it presented a number of optimizations that
could be used in a Haskell compiler as a cohesive document. It also showed,
albeit informally, that the execution times for the various optimizations
could make a dramatic difference, if the programmer has not carefully inserted
non-overloaded signatures.

\end{document}
