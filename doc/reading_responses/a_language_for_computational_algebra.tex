\documentclass[a4paper,fleqn,notitlepage]{scrartcl}

\title{Reading Response}
\subtitle{``A Language for Computational Algebra'' by Jenks and Trager}
\author{Ross Bayer}
\date{February 19, 2016}

\begin{document}

\maketitle

%%%%%%%%%%%%%%%%%%%%%%%%%%%%%%%%%%%%%%%%%%%%%%%%%%%%%%%%%%%%%%%%%%%%%%%%%%%%%%%%

Jenks and Trager outline their design for the development of a language
with extensible parameterized types and generic operators for computational
algebra in their paper \textit{``A Language for Computational Algebra''}. The
language they describe aims to provide programmers with tools to express
algorithms dealing with algebraic objects ``at their most natural level of
abstraction''. In an effort to facilitate the expression of algebraic properties,
the language provides a number of constructs, namely: \textit{categories},
\textit{domains}, and \textit{functors}.

A category, in the context of this paper, contains a set of generic operations
and a set of attributes that apply constraints to the concrete implementation
of said operators. More abstractly, a category is a grouping of domains that
share common operators and attributes.

Domains, in the context of this paper, are similar to the mathematical sense. A
domain describes a set of values that share a common set of generic operations
and more concretely, the implementations and characteristics of said operations.
For example, the authors point out the built-in domain \verb|Integer|, which is a 
member of the \verb|OrderedSet| category, which provides the \verb|<| operator.

Categories and Domains may be organized into hierarchies. Categories can be
created as extensions of previously defined categories, which will include
the extended operators and attributes as well as any that are newly defined.
For example, the built-in \verb|Set| category describes the operator \verb|=|, 
which is extended by the \verb|AbelianGroup| to provide the operators \verb|0|,
\verb|+|, \verb|-| in addition to the previously defined \verb|=|.

Functors are simply functions that return domains. A functor creates a domain,
which is a member of a category. Functors specify the the representation of the
domain. The domain remains a 

The language described in this paper features adhoc-polymorphism through the
\textit{categories}, \textit{domains} and \textit{functors} constructs. While
the language described by Jenks and Trager was clearly developed for computational
algebra, the implementation of adhoc-polymorphism can be applied to other general
purpose programming languages. It appears to me, that Jenks and Trager's paper
was a large influence on a number of more modern languages, such as Haskell and
Rust after it.

Haskell uses a similar approach to implementing adhoc-polymorphism. Categories 
appear analogous to Haskell's typeclasses. A typeclass defines a set of common
operations for a class of types. Domains and Functors are akin to concrete
instances of a typeclass, in which the definitions of the shared operations is
created for a given type. While the two approaches are similar, they still differ
in that categories also define shared attributes between members of the same class.
To my knowledge, a typeclass only defines shared operations.

\end{document}
