\documentclass[a4paper,fleqn,notitlepage]{scrartcl}

%\usepackage{float}
%\floatstyle{boxed} 
%\restylefloat{figure}

\title{Reading Response}
\subtitle{\normalfont ``A Second Look at Overloading'' by Martin Odersky, Philip Wadler, Martin Wehr}
\author{Ross Bayer}
\date{April 4, 2016}

\begin{document}

\maketitle

%%%%%%%%%%%%%%%%%%%%%%%%%%%%%%%%%%%%%%%%%%%%%%%%%%%%%%%%%%%%%%%%%%%%%%%%%%%%%%%%

``\textit{A Second Look at Overloading}'' by Martin Odersky, Philip Wadler and
Martin Wehr explores another strategy to implement adhoc polymorphism in a 
Hindley-Milner type system. The approach taken in the paper is contrasted with
that of Haskell's type classes.

This paper uses a small functional language called System O to demonstrate
operator overloading and the formal semantics, both dynamic and static. Rather
than using type classes to provide operator overloading by grouping related
over-loadable operators together under a name, System O takes a more simplistic
approach by allowing individual functions to be declared as over-loadable and
have instances for different types. This approach proves more flexible, in my
opinion than that of type classes. It could be used to implement type classes,
using a simple translation step during or directly after the parsing step of a
program, effectively as syntactic sugar.

One of the most serious criticisms of type classes, as used in Haskell, is their
inability to assign meaning to a program independent of it's types. It is not
possible to show a correspondence between the typed static semantics of a
program with type classes and the untyped, dynamic semantics. As the authors
point out, there exists no untyped dynamic semantics for type classes at all.
A simple restriction to type classes ensures that a program possesses a meaning
that can be determined independently of its type. For any type class over the
the type $\alpha$, each overloaded operator must have a type consistent with
the form $\alpha \rightarrow \tau$, where $\tau$ may contain $\alpha$. The
authors note that while this restriction makes it possible to construct an
untyped dynamic semantics, it is more limited in expression.

This paper focuses on a more generalized approach to type classes and operator
overloading. Rather than declaring that a set of operators belong to a class,
System O requires each operator to be individually declared as overloadable.
An operator that is overloadable can have multiple instance declarations for
different types. Types can still hold constraints over polymorphic types,
however those constraints are no longer based on type classes, but rather
individual operators. The loss of grouping with type classes may seem like a
detriment in terms of terseness and overall expressability, but as the authors
show with their example on page 137 (the example overloading $first$, $second$
and $third$), it can be more powerful.

The authors go to great lengths to prove the soundness of their extensions to
the typical Hindley/Milner type system. The typing rules are presented in a
clear manner with their additions clearly explained. What's more interesting is
the compositional semantics, which are accompanied by a proof of its soundness.

Much like the classical type classes, System O uses a translation step to
transform instance declarations into valid, well-typed programs conforming to the
underlying Hindley/Milner type system. The translation rules are straight-forward
using a similar ``dictionary-passing'' translation rule.

A really interesting feature of this paper is the inclusion of Record typing.
The authors show that records can be introduced into System O with minimal effort
and flexibly using overloading and another translation step.

Overall, I really enjoyed this paper's unique approach to operator overloading.
Maybe it's just the novelty of the solution, or the simplicity that makes it
so alluring to me. I find this solution more useful that type classes in Haskell
and I am left wanting a language that implements these ideas. I'm not sure if
any exists in the real world, or if the existence of such a language would be
receive the same kind of attention as those with type classes.

As a side note, after reading this paper, it has become clear to me that it would
be beneficial to read the initial paper(s) that introduce the Hindley/Milner type
system, as it is referenced throughout with the assumption that the reader is
familiar with its contents. I will tackle that paper(s) in my own time to better
understand the Hindley/Milner type system.

\end{document}
