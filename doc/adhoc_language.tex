\documentclass[11pt,a4paper]{article}

% imports
\usepackage[utf8]{inputenc}
\usepackage[english]{babel}

\usepackage{float}
\usepackage{hyperref}
\usepackage{listings}
% \usepackage{parskip}  % no-indent para
\usepackage{amsmath}
\usepackage{amssymb}  % number sets
\usepackage{syntax}   % (E)BNF

\usepackage{geometry}
\geometry{letterpaper,portrait,margin=1in}

% (E)BNF
% \usepackage{backnaur}
% \renewcommand{\bnfpo}{::=}

% judgements
\usepackage{semantic}
\reservestyle{\command}{\textbf}
\command{true,false}                                    % literals
\command{apply,closure,defun,defvar,if,lambda,let,prim} % keywords
\command{cons,nil}                                      % lists
\command{over,inst}                                     % over and inst


%%%%%%%%%%%%%%%%%%%%%%%%%%%%%%%%%%%%%%%%%%%%%%%%%%%%%%%%%%%%%%%%%%%%%%%%%%%%%%%%
% Title

\title{The \textit{Adhoc} Language}
\author{Ross Bayer}
\date{}

\begin{document}
\maketitle


%%%%%%%%%%%%%%%%%%%%%%%%%%%%%%%%%%%%%%%%%%%%%%%%%%%%%%%%%%%%%%%%%%%%%%%%%%%%%%%%
% Introduction

\section{Introduction}

\textit{Adhoc} is a small, functional programming language developed to study
adhoc polymorphism in Hindley-Milner type systems. The language is small and
was not developed as a general-purpose language. As such, it lacks many features
expected in a modern languge and should not be used non-trivial programs. Adhoc
is a funcitonal language that provides higher-order functions, recursive
functions, strong typing, parametric polymorphism and adhoc polymorphism.

This report defines the syntax and semantics of the Adhoc language. The
implementation details of adhoc polymorphism are explained in detail.


%%%%%%%%%%%%%%%%%%%%%%%%%%%%%%%%%%%%%%%%%%%%%%%%%%%%%%%%%%%%%%%%%%%%%%%%%%%%%%%%
% Syntax

\section{Syntax}

The syntax for the Adhoc language is heavily inspired by Scheme and Typed
Racket. The language features typical Lisp S-Expressions with a few select infix
operators for types.

%%%%%%%%%%%%%%%%%%%%%%%%%%%%%%%%%%%%%%%%%%%%%%%%%%%%%%%%%%%%%%%%%%%%%%%%%%%%%%%%
% Lexical Structure

\subsection{Lexical Structure}

The lexical structure of Adhoc is simple. The only valid characters in the
language are those representable in ASCII. Identifiers can contain a mixture of
special character and alpha-numeric. A more detailed lexical specification is
denoted below using Extended Bacus-Naur Form. It should be noted that
identifiers in the lanugage can contain a number of ``special'' characters,
including those typically associated with binary operators. This feature is
intentionally, so that users can define and overload typical operators.

% float styling
\floatstyle{boxed}
\restylefloat{figure}

\begin{figure}[H]
\small
\renewcommand{\grammarlabel}[2]{\synt{#1}\hfill#2}
\setlength{\grammarindent}{8em}
\begin{grammar}
<digit> ::= "0" ... "9"

<digits> ::= <digit>+

<int> ::= "-"? <digits>

<float> ::= "-"? <digits> ("." <digits>)?

<alpha> ::= "a" ... "z" | "A" ... "Z"

<special> ::= "!" | "\$" | "\&" | "+" | "-" | "*" | "/" | ":" | "<" | "=" | ">" | "?"| "~" | "_" | "^"

<initial> ::= <alpha> | <special>

<subsequent> ::= <alpha> | <digit> | <special> | "."

<ident> ::= <initial> <subsequent>+

<keywork> ::= "defun"
    \alt "defvar"
    \alt "if"
    % \alt "inst"
    \alt "lambda"
    \alt "let"
    % \alt "over"
    \alt "true"
    \alt "false"
\end{grammar}
\end{figure}


%%%%%%%%%%%%%%%%%%%%%%%%%%%%%%%%%%%%%%%%%%%%%%%%%%%%%%%%%%%%%%%%%%%%%%%%%%%%%%%%
% Grammar
\subsection{Grammar}

The grammar for the language \textbf{adhoc} is represented below in
Extended Backus-Naur Form. An interpreter implementing the language should
construct a parse tree from input and then transform the tree to a simplified
Abstract Syntax Tree which is then used in the type judgements and operational
semantics below. The Abstract Syntax Tree does not support multi-argument
function declarations with \verb|defun| or \verb|lambda|, but a simple
transformation can construct nested lambda expressions from a mult-argument
function declaration.

% float styling
\floatstyle{boxed}
\restylefloat{figure}

% grammar
\begin{figure}[H]
\small
\setlength{\grammarindent}{8em}
\renewcommand{\grammarlabel}[2]{\synt{#1}\hfill#2}
\begin{grammar}
<program> ::= <form> <program>

<form> ::= <definition>
    \alt <expression>
    % \alt <overload>
    % \alt <instance>

<definition> ::= "(" "defun" <ident> "(" ":" (<scheme>? "=>")? <formals> "->" <type> ")" <expression> ")"
    \alt "(" "defvar" <ident> <expression> ")"

% <overload> ::= "(" "over" <ident> ")"

% <instance> ::= "(" "inst" "(" <ident> ":" <type> ")" <expression> ")"

<expression> ::= <constant>
    \alt <ident>
    \alt "(" "if" <expression> <expression> <expression> ")"
    \alt "(" "lambda" <formals> <expression> ")"
    \alt "(" "let" <bindings> <expression> ")"
    \alt <application>
    \alt "(" <expression> ")"

<type> ::= "(" <type> ")"
    \alt <ident>
    \alt "(" <ident> ")"
    \alt <ident> "->" <type>

<scheme> ::= "(" ("forall" <ident>)? ")"

<variable> ::= <ident> ":" <type>

<formal> ::= "(" <variable> ")"

<formals> ::= "(" <variable>+ ")"

<value> ::= <ident> <expression>

<binding> ::= "(" <value> ")"

<bindings> ::= "(" <binding>+ ")"

<application> ::= "(" <expression> <expression>+ ")"
\end{grammar}
\end{figure}


%%%%%%%%%%%%%%%%%%%%%%%%%%%%%%%%%%%%%%%%%%%%%%%%%%%%%%%%%%%%%%%%%%%%%%%%%%%%%%%%
% Type System

\section{Type Judgements}

Adhoc features a Hindley-Milner type system with parametric polymorphism.
Monotypes $\tau$ designate a concrete type, such as \verb|bool|, \verb|float|,
\verb|int| and also parametric types, such as \verb|list| and $\rightarrow$.
Polytypes, also called type schemes, allow for universal quantification over
a type variable $\alpha$ in another polytype or monotype.

% \begin{figure}[H]
% \centering
% \caption{Variables}
% \label{fig:vars}
% % \renewcommand{\arraystretch}{1.5}
% \begin{tabular}{l l}
%     Unique variables        & $u \in \mathcal{U}$ \\
%     Overloaded variables    & $o \in \mathcal{O}$ \\
%     Variables               & $x = u$ \| $o$         \\

%     Type variables          & $\alpha \in \mathcal{A}$ \\
%     Type constructors       & $T \in \mathcal{T}$   \\
%     Type schemes            & $\tau = \alpha$ | $\forall \alpha . \pi_\alpha \Rightarrow \sigma$ \\
%     Constraints on $\alpha$ & $\pi_\alpha = o_1 : \alpha \rightarrow \tau_1$, \dots, $o_n : \alpha \rightarrow \tau_n$ \\
%     Typotheses              & $\Gamma = x_1 : \sigma_1, \dots, x_n : \sigma_n$ \\
% \end{tabular}
% \end{figure}

% float styling
\floatstyle{boxed}
\restylefloat{figure}

\begin{figure}[H]
\small
\setlength{\grammarindent}{10em}
\begin{grammar}
<monotype $\tau$> = "bool"
    \alt "float"
    \alt "int"
    \alt "list" $\tau$
    \alt $\tau_1 \rightarrow \tau_2$

<polytype $\sigma$> = $\tau$
    \alt $\forall$ $\alpha$ . $\sigma$
\end{grammar}
\end{figure}

Literals in the lanugage have their types inferred using the following
judgements.

\begin{figure}[H]
\centering
% literals
\[
    % bool
    \inference[BOOL ]{}{\Gamma |- \<true> : bool}
    \hspace{1em}
    \inference{}{\Gamma |- \<false> : bool}[BOOL]
\]
\[
    % int
    \inference[INT]{n \in \mathbb{Z}}{\Gamma |- n : int}
    \hspace{1em}
    % float
    \inference{n \in \mathbb{R}}{\Gamma |- n : float}[FLOAT]
\]
\[
    % nil
    \inference[NIL]{}{\Gamma |- \<nil> : list_\tau}
    \hspace{1em}
    % cons
    \inference
        {
            \Gamma |- M : \tau \\
            \Gamma |- N : \tau
        }
        {\Gamma |- \<cons>(M, N) : list_\tau}
        [CONS]
\]
\end{figure}

Definitions and expressions in the lanague have their types inferred using the
following judgements.

\begin{figure}[H]
\centering
\small
\[
    % variable
    \inference[TAUT]{}{\Gamma |- x : \sigma}
    \hspace{2em}
    % if
    \inference
        {
            \Gamma |- M_1 : bool \\
            \Gamma |- M_2 : \tau \\
            \Gamma |- M_3 : \tau
        }
        {\Gamma |- \<if>(M_1, M_2, M_3) : \tau}
        [IF]
\]

\[
    % let
    \inference[LET]
        {
            \Gamma |- M : \tau \\
            \Gamma, M : \tau |- N : \tau'
        }
        {\Gamma |- \<let>(x, M, N) : \tau'}
    \hspace{2em}
    % lambda
    \inference
        {\Gamma, x : \tau |- M : \tau'}
        {\Gamma |- \<lambda>(x, \tau, N) : \tau'}
        [LAMBDA]
\]

\[
    % defun
    \inference[DEFUN]
        {\Gamma, x : \forall \alpha . \tau, M : \tau' |- f : \tau -> \tau'}
        {\Gamma |- \<defun>(f, x, \tau, M, \tau') : \tau -> \tau'}
    \hspace{2em}
    % defvar
    \inference
        {\Gamma |- M : \tau}
        {\Gamma |- \<defvar>(x, M) : \tau}
        [DEFVAR]
\]

\[
    % apply
    \inference[APPLY]
        {
            \Gamma |- N : \tau \\
            \Gamma |- M : \tau -> \tau'
        }
        {\Gamma |- \<apply>(M, N) : \tau'}
\]
\end{figure}


%%%%%%%%%%%%%%%%%%%%%%%%%%%%%%%%%%%%%%%%%%%%%%%%%%%%%%%%%%%%%%%%%%%%%%%%%%%%%%%%
% Semantics

\section{Semantics}

The \textbf{adhoc} langauge has clearly defined operational semantics.
To allow for self-recursive functions a global environment $G$ is
threaded through. The operational semantics for literals is as follows:

\begin{figure}[H]
\centering
\small
% boolean literals

\[
    \inference[BOOL]{}{(\<true>, \rho, G) \Downarrow (\<true>, G)}
    \hspace{1em}
    \inference{}{(\<false>, \rho, G) \Downarrow (\<false>, G)}[BOOL]
\]

\[
    % int
    \inference[INT]{n \in \mathbb{Z}}{(n, \rho, G) \Downarrow (n, G)}
    \hspace{1em}
    % float
    \inference{n \in \mathbb{R}}{(n, \rho, G) \Downarrow (n, G)}[FLOAT]
\]

% variables
\[\inference[VAR]{$lookup$(x, G) = v}{(x, \rho, G) \Downarrow (v, G')}\]

% nil
\[
    \inference[NIL]{}{(\<nil>(\tau), \rho, G) \Downarrow (\<nil>(\tau), G)}
    \hspace{1em}
    % cons
    \inference
        {
            (M, \rho, G) \Downarrow (v', G') \\
            (N, \rho, G) \Downarrow (v'', G'')
        }
        {(\<cons>(M, N), \rho, G) \Downarrow (\<cons>(v', v''), G)}
        [CONS]
\]
\end{figure}


The operational semantics for more complex expressions is as follows:

\begin{figure}[H]
\centering
\small
\[
    % if
    \inference
        [IF]
        {
            (M_1, \rho, G) \Downarrow (\<true>, G) \\
            (M_2, \rho, G') \Downarrow (v, G'')
        }
        {(\<if>(M_1, M_2, M_3), \rho, G) \Downarrow (v, G'')}
    \hspace{1em}
    \inference
        {
            (M_1, \rho, G) \Downarrow (\<false>, G) \\
            (M_3, \rho, G') \Downarrow (v, G'')
        }
        {(\<if>(M_1, M_2, M_3), \rho, G) \Downarrow (v, G'')}
        [IF]
\]

\[
    % let
    \inference
        [LET]
        {
            (M_1, \rho, G) \Downarrow (v, G') \\
            (M_2, \rho[x \mapsto v], G') \Downarrow (v', G'')
        }
        {(\<let>(x, M_1, M_2), \rho, G) \Downarrow (v', G'')}
\]

\[
    % lambda
    \inference
        [LAMBDA]
        {}
        {(\<lambda>(x, t_x, M), \rho, G) \Downarrow (\<closure>(x, M, \rho), G)}
\]

\[
    % defun
    \inference
        [DEFUN]
        {}
        {(\<defun>(f, x, t_x, M, t_M), \rho, G) \Downarrow ($_$, \rho, G[f \mapsto \<closure>(x, M, \rho)])}
\]

\[
    % defvar
    \inference
        [DEFVAR]
        {}
        {(\<defvar>(x, M), \rho, G) \Downarrow ($_$, \rho, G[x \mapsto M])}
\]

\[
    % apply closure
    \inference
        [APPLY]
        {
            (M, \rho, G) \Downarrow (\<closure>(x, M_c,\rho'), G') \\
            (N, \rho, G') \Downarrow (v, \rho, G'') \\
            (M_c, \rho'[x \mapsto v], G'') \Downarrow (v', G''')
        }
        {(\<apply>(M, N), \rho, G) \Downarrow (v', G''')}
    \hspace{1em}
    % apply primitive
    \inference
        {
            (M, \rho, G) \Downarrow (\<prim>(f), G') \\
            (N, \rho, G') \Downarrow (v, \rho, G'') \\
            \delta(f, v) = v'
        }
        {(\<apply>(M, N), \rho, G) \Downarrow (v', G'')}
        [APPLY]
\]
\end{figure}

%%%%%%%%%%%%%%%%%%%%%%%%%%%%%%%%%%%%%%%%%%%%%%%%%%%%%%%%%%%%%%%%%%%%%%%%%%%%%%%%
% Extensions for Adhoc-Polymorphism

\section{Extensions for Adhoc-Polymorphism}

The Adhoc language bases it's implementation of adhoc polymorphism on the paper
\textit{``A second look at overloading''} by Martin Odersky, Philip Wadler and
Martin Wehr. To learn more about the rationale for this more granular approach
to adhoc polymorphism over type classes, read the paper.

To extend the base language to support adhoc polymorphism, two new constructs
must be introduced, overload declarations and instance declarations. Overload
declarations allow users to declare an operator or function identifier as
overloadable for the duration of the program. Instance declarations allow users
to declare an instance for an overloadable operator or function with an explicit
type and possible constraints, which maps to a concrete expression.

Type schemes must be changed to allow for constraints, represented as $\pi_\alpha$,
which translates to a list of overloaded operators and their types, which must be
of the form $\alpha \rightarrow \tau$, where $\alpha$ is the same for all
operators and $\tau$ can contain uses of $\alpha$.

\[\pi_\alpha = o_1 : \alpha \rightarrow \tau_1, \dots, o_n : \alpha \rightarrow \tau_n\]

Types are now of the form

\begin{figure}[H]
\small
\begin{flalign*}
    \tau & = \text{bool} \\
         & = \text{float} \\
         & = \text{int} \\
         & = \text{list}_\tau \\
         & = \tau \rightarrow \tau
\end{flalign*}
\begin{flalign*}
    \sigma & = \tau \\
           & = \forall \alpha . \pi_\alpha \Rightarrow \sigma
\end{flalign*}
\end{figure}

The syntax for the language changes a bit, the \verb|<scheme>| production rule
allows for type constraints and two new types of declarations are introduced for
\verb|over| and \verb|inst| declarations.

\begin{figure}[H]
\small
\setlength{\grammarindent}{8em}
\renewcommand{\grammarlabel}[2]{\synt{#1}\hfill#2}
\begin{grammar}
<form> ::= <definition>
    \alt <expression>
    \alt <overload>
    \alt <instance>

<overload> ::= "(" "over" <ident> ")"

<instance> ::= "(" "inst" <ident> "(" ":" (<scheme>? "=>")? <type> ")" <expression> ")"

<scheme> ::= "(" "forall" <ident> <constraint>?")"

<constraint> ::= "(" <ident> ":" <type> ")"
\end{grammar}
\end{figure}

% The type environment must also be changed into a 3 pair containing the typothesis
% $\Gamma$, a mapping of overloaded instances and a mapping of instance names to
% unique identifier. When an \verb|over| declaration is evaluated, it should be
% added to the type environment, if that operator has not already been declared as
% overloaded. When an \verb|inst| declarationis evaluated, it should be entered
% into the mapping from overloaded operator to instances and a new, unique identifier
% should

Lastly, the type judgements are modified to not only infer and enforce types,
but also transform the Abstract Syntax Tree so as to eliminate uses of \verb|inst|
declarations and constraints. The inference rules for literals and the operational
semantics remain unchanged. Below is the modified type judgements and transformations.
All unlisted type judgements are unchanged and transform into themselves.

\begin{figure}[H]
\centering
\small
\[
    % lambda
    \inference
        [LAMBDA]
        {
            \Gamma, x : \tau |- M : \tau'
        }
        {
            \Gamma |- \<lambda>(x, \tau, N) : \tau'
            \succ \<lambda>(x, \tau, N)
        }
\]

\[
    % defun
    \inference
        [DEFUN]
        {
            \Gamma, x : \forall \alpha . \tau, M : \tau' |- f : \tau -> \tau'
        }
        {
            \Gamma |- \<defun>(f, x, \tau, M, \tau') : \tau -> \tau'
            \succ \<defun>(f, x, \tau, M, \tau')
        }
\]

\[
    % forall introduction
    \inference
        [($\forall$I)]
        {
            \Gamma, (o_1 : \alpha \rightarrow \tau_1, \dots, o_n : \alpha \rightarrow \tau_n) |- e : \sigma \succ e^{*}
        }
        {
            \Gamma |- e : \forall \alpha . (o_1 : \alpha \rightarrow \tau_1, \dots, o_n : \alpha \rightarrow \tau_n) \Rightarrow \sigma
            \succ \<lambda>(x_{\tau_1}, \ldots, \<lambda>(x_{\tau_n}, e^{*}))
        }
\]

\[
    % forall elimination
    \inference
        [($\forall$E)]
        {
            \Gamma |- e : \forall \alpha . (o_1 : \alpha \rightarrow \tau_1, \dots, o_n : \alpha \rightarrow \tau_n) |- e : \sigma \succ e^{*} \\
            \Gamma |- o_i : [\tau/\alpha]\tau_i \succ e_{i}^{*} (i = 1, \ldots, n)
        }
        {
            \Gamma |- e : [\tau/\alpha]\sigma
            \succ \<apply>(e^{*}, \<apply>(e_{1}^{*}, \ldots, e_{n}^{*}))
        }
\]

\[
    % inst
    \inference
        [INST]
        {
            o : \sigma_{\tau'} \in \Gamma \Rightarrow \tau \neq \tau' \\
            \Gamma |- e : \sigma_{\tau} \succ e^{*}
        }
        {
            \Gamma |- \<inst>(o, \sigma_{\tau}, e)
            \succ \<defvar>(o,u_{\sigma_{\tau}}, e^{*})
        }
\]
\end{figure}

The above type judgements present the typical dictionary translation, which
transforms unique instances of overloaded operators into unique variable
bindings. The name of each is dependent on the operator and the type signature.
All type constraints associated with an expression are transformed into nested
lambdas that take arguments for each overloaded operator in the constraints.

\end{document}
