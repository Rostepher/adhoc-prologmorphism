\documentclass[10pt,a4paper,fleqn]{article}

% imports
\usepackage[utf8]{inputenc}
\usepackage[english]{babel}

\usepackage{float}
\usepackage{hyperref}
\usepackage{listings}
% \usepackage{parskip}  % no-indent para
\usepackage{amssymb}  % number sets
\usepackage{syntax}   % (E)BNF

% (E)BNF
\usepackage{backnaur}
\renewcommand{\bnfpo}{::=}

% judgements
\usepackage{semantic}
\reservestyle{\command}{\textbf}
\command{true,false}                                    % literals
\command{apply,closure,defun,defvar,if,lambda,let,prim} % keywords
\command{cons,nil}                                      % lists


%%%%%%%%%%%%%%%%%%%%%%%%%%%%%%%%%%%%%%%%%%%%%%%%%%%%%%%%%%%%%%%%%%%%%%%%%%%%%%%%
% Title & TOC

\title{The \textit{adhoc} Language}
\author{Ross Bayer}
\date{\today}

\begin{document}
\maketitle


%%%%%%%%%%%%%%%%%%%%%%%%%%%%%%%%%%%%%%%%%%%%%%%%%%%%%%%%%%%%%%%%%%%%%%%%%%%%%%%%
% Abstract

\section*{Abstract}

This report defines the syntax, typing and semtics for the small programming
language \textit{adhoc}.


%%%%%%%%%%%%%%%%%%%%%%%%%%%%%%%%%%%%%%%%%%%%%%%%%%%%%%%%%%%%%%%%%%%%%%%%%%%%%%%%
% Introduction

\section{Introduction}

Adhoc polymorphism is ...

The language \textit{adhoc} is a small, functional programming language that ...

Do not use it for non-trivial programs, as the language is not intended as a
general-purpose language, but rather as a proof-of-concept and learning
project.


%%%%%%%%%%%%%%%%%%%%%%%%%%%%%%%%%%%%%%%%%%%%%%%%%%%%%%%%%%%%%%%%%%%%%%%%%%%%%%%%
% Lexical Structure

\section{Lexical Structure}

The lexical structure of the language \textbf{adhoc} is simple. The only
valid characters are those representable in ASCII. Identifiers can contain
a mixture of special character and alpha-numeric. The lexical specification
is denoted using EBNF form.

% float styling
\floatstyle{boxed}
\restylefloat{figure}

\begin{figure}[H]
\renewcommand{\grammarlabel}[2]{\synt{#1}\hfill#2}
\setlength{\grammarindent}{8em}
\begin{grammar}
<digit> ::= 0 | 1 | ... | 9

<digits> ::= <digit>+

<int> ::= '-'? <digits>

<float> ::= '-'? <digits> ('.' <digits>)?

<alpha> ::= a | b | ... | z | A | B | ... | Z

<initial> ::= <alpha> | <special>

<subsequent> ::= <alpha> | <digit> | <special> | '.'

<ident> ::= <initial> <subsequent>+

<keywork> ::= 'defun'
    \alt 'defvar'
    \alt 'if'
    \alt 'lambda'
    \alt 'let'
    \alt 'true'
    \alt 'false'

\end{grammar}
\end{figure}

\begin{figure}[H]
    \begin{bnf*}
        \bnfprod{list}
                {\bnfpn{listitems} \bnfor \bnfes} \\
        \bnfprod{listitems}
                {\bnfpn{item} \bnfor \bnfpn{items}
                 \bnfsp \bnfts{;} \bnfsp \bnfpn{listitems}} \\
        \bnfprod{item}
                {\bnftd{description of item}}
    \end{bnf*}
\end{figure}


%%%%%%%%%%%%%%%%%%%%%%%%%%%%%%%%%%%%%%%%%%%%%%%%%%%%%%%%%%%%%%%%%%%%%%%%%%%%%%%%
% Grammar
\section{Grammar}

The grammar for the language \textbf{adhoc} is represented below in
EBNF form. It translates to a simplified AST used in the type judgements
and operational semantics below.

% float styling
\floatstyle{boxed}
\restylefloat{figure}

% grammar
\begin{figure}[H]
\setlength{\grammarindent}{8em}
\renewcommand{\grammarlabel}[2]{\synt{#1}\hfill#2}
\begin{grammar}
<program> ::= <form> <program>

<form> ::= <definition>
    \alt <expression>

<definition> ::= '(' 'defun' <ident> ':' <type> <expression> ')'
    \alt '(' 'defvar' <ident> <expression> ')'

<expression> ::= <constant>
    \alt <ident>
    \alt '(' 'if' <expression> <expression> <expression> ')'
    \alt '(' 'lambda' <formals> <expression> ')'
    \alt '(' 'let' <bindings> <expression> ')'
    \alt <application>
    \alt '(' <expression> ')'

<variable> ::= <ident> ':' <type>

<type> ::= '(' <type> ')'
    \alt <ident>
    \alt '(' <ident> ')'
    \alt <ident> '->' <type>

<formal> ::= '(' <variable>+ ')'

<value> ::= <ident> <expression>

<bindings> ::= '(' <value>+ ')'

<application> ::= '(' <expression> <expression>+ ')'
\end{grammar}
\end{figure}


%%%%%%%%%%%%%%%%%%%%%%%%%%%%%%%%%%%%%%%%%%%%%%%%%%%%%%%%%%%%%%%%%%%%%%%%%%%%%%%
% Type Judgements

\section{Type Judgements}

The language features strong typing with the following built-in types:

% float styling
\floatstyle{boxed}
\restylefloat{figure}
\begin{figure}[H]
\setlength{\grammarindent}{5em}
\begin{grammar}
<type> ::= int
       \alt float
       \alt bool
       \alt list $t$
       \alt $t_1 \rightarrow t_2$
\end{grammar}
\end{figure}

Below are the type judgements used to infer and validate the types of
expressions in the language.

% bool literals
\[
    \inference{}{\Gamma |- \<true> : bool}
    \hspace{1em}
    \inference{}{\Gamma |- \<false> : bool}
\]

% int and float
\[
    \inference{n \in \mathbb{Z}}{\Gamma |- n : int}
    \hspace{1em}
    \inference{n \in \mathbb{R}}{\Gamma |- n : float}
\]

% variable
\[\inference{x : \tau \in \Gamma}{\Gamma |- x : \tau}\]

% if
\[\inference
	{
    	\Gamma |- M_1 : bool \\
    	\Gamma |- M_2 : \tau \\
        \Gamma |- M_3 : \tau
    }
    {\Gamma |- \<if>(M_1, M_2, M_3) : \tau}
\]

% let
\[\inference
	{
	    \Gamma |- M : \tau \\
	    \Gamma, M : \tau |- N : \tau'
	}
    {\Gamma |- \<let>(x, M, N) : \tau'}
\]

% lambda
\[\inference
    {\Gamma, x : \tau |- M : \tau'}
    {\Gamma |- \<lambda>(x, \tau, N) : \tau'}
\]

% define
\[\inference
    {\Gamma, x : \tau, M : \tau' |- f : \tau -> \tau'}
    {\Gamma |- \<defun>(f, x, \tau, M, \tau') : \tau -> \tau'}
\]

% const
\[\inference
    {\Gamma |- M : \tau}
    {\Gamma |- \<defvar>(x, M) : \tau}
\]

% call
\[\inference
    {
        \Gamma |- N : \tau \\
        \Gamma |- M : \tau -> \tau'
    }
    {\Gamma |- \<apply>(M, N) : \tau'}
\]

% nil
\[\inference{}{\Gamma |- \<nil> : list_\tau}\]

% cons
\[\inference
    {
        \Gamma |- M : \tau \\
        \Gamma |- N : \tau
    }
    {\Gamma |- \<cons>(M, N) : list_\tau}
\]


%%%%%%%%%%%%%%%%%%%%%%%%%%%%%%%%%%%%%%%%%%%%%%%%%%%%%%%%%%%%%%%%%%%%%%%%%%%%%%%%
% Operational Semantics

\section{Operational Semantics}

The \textbf{adhoc} langauge has clearly defined operational semantics.
To allow for self-recursive functions a global environment $G$ is
threaded through. The operational semantics for the language are as
follows:

% boolean literals

\[
    \inference{}{(\<true>, \rho, G) \Downarrow (\<true>, G)}
    \hspace{1em}
    \inference{}{(\<false>, \rho, G) \Downarrow (\<false>, G)}
\]

% numbers
\[\inference{}{(n, \rho, G) \Downarrow (n, G')}\]

% variables
\[\inference{$lookup$(x, G) = v}{(x, \rho, G) \Downarrow (v, G')}\]

% if
\[
\inference
	{
 	    (M_1, \rho, G) \Downarrow (\<true>, G) \\
        (M_2, \rho, G') \Downarrow (v, G'')
    }
	{(\<if>(M_1, M_2, M_3), \rho, G) \Downarrow (v, G'')}
\hspace{1em}
\inference
	{
 	    (M_1, \rho, G) \Downarrow (\<false>, G) \\
        (M_3, \rho, G') \Downarrow (v, G'')
	}
	{(\<if>(M_1, M_2, M_3), \rho, G) \Downarrow (v, G'')}
\]

% let
\[\inference
 	{
     	(M_1, \rho, G) \Downarrow (v, G') \\
        (M_2, \rho[x \mapsto v], G') \Downarrow (v', G'')
    }
 	{(\<let>(x, M_1, M_2), \rho, G) \Downarrow (v', G'')}
\]

% lambda
\[\inference
 	{}
    {(\<lambda>(x, t_x, M), \rho, G) \Downarrow (\<closure>(x, M, \rho), G)}
\]

% defun
\[\inference
 	{}
 	{(\<defun>(f, x, t_x, M, t_M), \rho, G) \Downarrow ($_$, \rho, G[f \mapsto \<closure>(x, M, \rho)])}
\]

% defvar
\[\inference
 	{}
    {(\<defvar>(x, M), \rho, G) \Downarrow ($_$, \rho, G[x \mapsto M])}
\]

% call closure
\[\inference
 	{
     	(M, \rho, G) \Downarrow (\<closure>(x, M_c,\rho'), G') \\
      	(N, \rho, G') \Downarrow (v, \rho, G'') \\
      	(M_c, \rho'[x \mapsto v], G'') \Downarrow (v', G''')
     }
     {(\<apply>(M, N), \rho, G) \Downarrow (v', G''')}
\]

% call primitive
\[\inference
 	{
     	(M, \rho, G) \Downarrow (\<prim>(f), G') \\
      	(N, \rho, G') \Downarrow (v, \rho, G'') \\
  	    \delta(f, v) = v'
     }
 {(\<apply>(M, N), \rho, G) \Downarrow (v', G'')}
\]

% nil
\[\inference{}{(\<nil>(\tau), \rho, G) \Downarrow (\<nil>(\tau), G)}\]

% cons
\[\inference
    {
        (M, \rho, G) \Downarrow (v', G') \\
        (N, \rho, G) \Downarrow (v'', G'')
    }
    {(\<cons>(M, N), \rho, G) \Downarrow (\<cons>(v', v''), G)}
\]

%%%%%%%%%%%%%%%%%%%%%%%%%%%%%%%%%%%%%%%%%%%%%%%%%%%%%%%%%%%%%%%%%%%%%%%%%%%%%%%%
% Extensions for Adhoc-Polymorphism

\section{Extensions for Adhoc-Polymorphism}

TODO


%%%%%%%%%%%%%%%%%%%%%%%%%%%%%%%%%%%%%%%%%%%%%%%%%%%%%%%%%%%%%%%%%%%%%%%%%%%%%%%%
% Standard Library

\section{Standard Library}

TODO

\end{document}
